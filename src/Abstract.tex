\begin{abstract}

Continuous bio-signal sensing applications require circuits with low power consumption and small die area. This creates a need for optimally designed sensing amplifiers balancing noise, power consumption and circuit area. Capacitively coupled instrumentation amplifiers (CCIAs) with chopping and amplifiers biased in weak inversion are a popular design choice for realising low noise amplifiers (LNAs) for bio-signal amplification.

In this design overview paper we introduce our open-source Python based \gmID design tool that enables the fast realisation of optimised bio-signal LNAs. The design uses Jupyter Notebook, facilitating accessible, rapid design and trade-off analysis. A design methodology for realising low noise CCIAs is presented. The trade-off between \gmID and input-referred noise (IRN) is explored, highlighting the effect of large device sizes in weak-inversion. Trade-offs between circuit area and power consumption for area constrained bio-sensor circuits, especially in the neural sensing domain, are presented.

To demonstrate the efficacy of the design methodology a ultra-low noise LNA has been designed using a 65nm technology. 
The designed circuit is presented with measured chip results demonstrating \SI{2.07}{\nano\volt\per\sqrt{\hertz}} in-band noise.
\end{abstract}