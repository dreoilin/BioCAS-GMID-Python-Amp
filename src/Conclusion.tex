\section{Conclusion}\label{sec:conclusion}
The design of bio-sensor circuits, particularly those employing CCIAs, requires careful balancing of area, noise, and power consumption. 
%The need to minimize flicker noise often leads to the use of weak inversion, which in turn can result in increased area due to the exponential growth of transistor size and input capacitance. By adopting a systematic sizing approach and careful frequency planning, it is possible to achieve optimal designs in moderate inversion, enabling compact footprints suitable for multi-channel or \textit{in-vivo} applications. 
This paper has introduced a robust and systematic design procedure for bio-sensing amplifiers, supported by an enhanced open-source Python \gmID toolkit. The methodology, implemented through a Jupyter Notebook, provides an accessible and efficient means for designers to perform rapid design and trade-off analysis. The effectiveness of this approach has been demonstrated through the successful design of an ultra-low noise CCIA with a noise figure of \SI{2.07}{\nano\volt\per\sqrt{\hertz}}. The \gmID design methodology and the accompanying Python toolkit presented in this work offer a valuable framework for future designers, enabling them to quickly visualise and refine their amplifier designs, thereby facilitating the development of optimised bio-sensing circuits with minimal footprint and enhanced performance. All design plots used in this paper can be recreated by characterising a technology with PyGMID \cite{O_Donnell_PyGMID} and using the Jupyter Notebook in this dedicated repository \cite{CCIA_GMID}.






